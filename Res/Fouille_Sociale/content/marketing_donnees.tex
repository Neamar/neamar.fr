La fouille de données des réseaux sociaux représente une mine d'or d'informations qui fait l'objet d'un véritable marketing. Dans cette partie nous allons nous pencher sur l'utilisation de ces informations par des entreprises dans le but de prospecter de nouveaux clients ou de cibler les entités influentes afin d'introduire une stratégie marketing dite "virale".

\subsection{Prospection de nouveaux clients}
Une des principales applications du datamining est d'aider les entreprises à déterminer qui est susceptible de représenter un client potentiel, et qui n'est pas enclin à acheter son produit. Cette fouille de données peut jouer un rôle crucial dans la stratégie commerciale et marketing d'une entreprise. Il existe deux principaux types de démarchages qui sont le \textbf{marketing de masse} et le \textbf{marketing direct} \cite{MarketMasseDirect}.\\
Le premier vise un public large appartenant à un même segment\footnote{Un segment marketing (ou de clientèle) est une catégorie identifiable de clients visés.} via des médias de masse comme la télévision, le radio ou les journaux. C'est une méthode très efficace quand un produit est très demandé par les clients, comme par exemple l'électroménager après la seconde guerre mondiale. Aujourd'hui avec l'abondance de produits et la forte concurrence, les moeurs ont changé et ce type de marketing est jugé peu efficace.\\
Le second type de marketing, le marketing direct, cible un public à fort potentiel d'achat sur un produit spécifique. Ce public est visé en se basant seulement sur ses caractéristiques et ses besoins. La fouille de données joue un rôle clé dans le marketing direct puisqu'il permet de donner un modèle prévisionnel du comportement d'un client en fonction de ses derniers achats (habitudes de consommation, mode de paiement...) et d'informations personnelles (situation géographique, centres d'intérêts...).

Limiter ainsi sa cible à des clients ayant un fort potentiel d'achat est ce que l'on appelle en marketing le "scoring client". La prospection coûte cher aux entreprises, c'est pourquoi il leur est primordial de limiter le nombre de cibles à démarcher et de viser seulement les individus jugés les plus susceptibles d'acheter leurs produits. Avec l'arrivée d'Internet et des réseaux sociaux, les informations personnelles sur les individus sont beaucoup plus riches et facilement accessibles. Il est possible d'extrapoler ces informations et d'en déduire le comportement et les habitudes de consommation d'un individu. Ces informations une fois recoupées et analysées vont permettre d'assimiler ses individus à une niche de clients. Chaque niche est susceptible de présenter un fort potentiel d'achat à l'égard de certains produits et d'être réticent à d'autres.

Ainsi le datamining s'avère très efficace et donc très utilisé dans le marketing direct pour cibler un public à même d'être intéressé par le produit publicisé. Cependant, le datamining ne doit pas être considéré comme une solution miracle à l'ensemble des problèmes des entreprises. Il correspond à une avancée technologique qui permet de faire face au volume croissant des données collectées mais les entreprises devront instaurer un climat de confiance afin de ne pas porter atteinte à la vie privée des clients en exploitant les données collectées.

Les premières applications de datamining concernaient l'étude des tickets de caisse des clients de grande surface. Cela a permis de montrer que pour certaines catégories de clients les promotions mises en place pour des produits qu'ils avaient l'habitude d'acheter simultanément n'étaient pas efficaces et n'engendraient pas d'augmentation de chiffre d'affaires.

D'après les études de l'université de Western Ontario \cite{kdd98}, la démarche pour cibler un public présentant un fort potentiel d'achat pour un produit spécifique est toujours similaire à celle-ci :
\begin{enumerate}
  \li Récupérer les données de tous les clients possibles (retours des dernières prospections, informations personnelles sur des individus...)
  \li Dataminer à cette base de données, c'est-à-dire :
    \item ajouter des informations géo-démographiques
    \item traiter les valeurs manquantes (déduites des autres données complètes)
    \item séparer la base de données en deux jeux, qu'on appelle le "testing set" et "training set"
    \item appliquer les algorithmes au "training set" pour estimer leur comportement en fonction des analyses déduites de l'étude du "testing set"
    \li Évaluer le modèle déduit du "testing set". Recommencer la deuxième étape s'il n'est pas satisfaisant.
    \li Utiliser le modèle déduit pour séparer les clients potentiels des non-acheteurs.
    \li Promouvoir le produit aux clients potentiels retenus.
\end{enumerate}

Cette démarche théorique et mathématique a fait ses preuves dans la pratique. Elle permet réellement de rentabiliser les opérations de prospection, en diminuant les coûts engendrés par l'acquisition de nouveaux clients. En mettant en place une solution de datamining, les entreprises essaient d'allonger la "durée de vie" d'un client en repérant les raisons  et les risques de son départ. Par exemple, les sociétés de vente par correspondance réalisent des catalogues spécialisés en plus de leur catalogue généraliste. L'utilisation du datamining permet de sélectionner parmi les clients principaux ceux pour lesquels il est utile d'envoyer un catalogue spécialisé. C'est en effet grâce à l'historique des achats qu'il est possible de déterminer quel client est susceptible d'acheter un article sur catalogue spécialisé \cite{MM}.

Cependant l'une des principales limites à cette démarche théorique vient du fait que la décision d'achat d'un client est considérée comme indépendante des autres alors qu'en réalité, cette décision est grandement influencée par ses collègues, ses amis, son entourage... Négliger l'influence d'un membre du réseau sur un autre est donc une mauvaise estimation du comportement d'un réseau puisque c'en est presque l'un des fondements. Cette influence est d'ailleurs ajourd'hui utilisée à des fins commerciales dans des campagnes marketing dites "virales".


\subsection{Marketing viral}

Comme nous venons de le voir, ignorer les effets de réseaux quand on décide quels clients démarcher peut mener à des décisions peu optimales. Le département informatique de l'université de Washington a étudié ce sujet et propose la modélisation d'une \textbf{valeur de réseau} \cite{kdd01a}, qui évalue l'influence qu'un client a sur les autres. En plus de la valeur intrinsèque du profit potentiel qu'un client peut représenter, cette étude propose donc d'ajouter une valeur qui représente l'importance de son réseau et donc son influence sur les autres clients. Ainsi un client potentiel représentant peu de profit mais étant très influent mériterait tout de même d'être démarché. Inversement, il serait redondant de démarcher un client avec un profit potentiel élevé puisqu'il a de fortes chances d'être influencé par son réseau. C'est le fondement de ce qui est appelé le \textbf{marketing viral}. Quand cette approche fonctionne, cela permet à l'entreprise d'augmenter significativement ses profits et de limiter son budget de communication et de publicité.

Le marketing viral est basé sur le réseau du "bouche à oreille" et propose un bien meilleur rapport qualité/prix qu'un marketing conventionnel puisqu'il s'appuie sur les clients eux-mêmes pour faire la promotion d'un produit. C'est là que réside l'un des éléments essentiels du marketing viral : chaque client devient un vendeur involontaire, simplement en utilisant le produit. Il est plus puissant que la publicité classique car il véhicule l'approbation implicite d'un ami. Un élément clé de la consommation de marque est d'ailleurs l'affiliation à un groupe : "je veux être un membre du groupe, le groupe étant dans ce cas les amis qui utilisent ledit produit". Ce type de marketing est essentiellement opéré via Internet, notamment sur les réseaux sociaux, et non les autres médias plus classiques auxquels les individus deviennent de plus en plus réticents. Le marketing viral exploite donc ces réseaux sociaux pour encourager les clients à partager des informations sur leurs produits. Ce type de marketing n'est donc pas efficace sur tous les types produits et il est difficile de mesurer son impact.

Un exemple des plus éloquents, développé dans le magazine Red Herring \cite{VM}, est celui du service de courriers électroniques "Free Hotmail" qui est passé de zéro à douze millions d'utilisateurs en moins d'un an et demi avec un budget publicitaire minime (cinquante mille dollars\footnote{À la même époque, la compagnie Juno a dépensé plus de vingt millions de dollars sur une campagne conventionnelle pour un effet moindre.}), et ce grâce aux messages promotionnels inclus dans le pied de page de chaque email envoyé depuis leur service.

L'université de Canergie Mellon propose un modèle pour identifier quels types de produits sont susceptibles d'être promus par le marketing viral \cite{viralTWeb}. Ce modèle est basé sur quelques principes évidents mais dont l'étude a levé quelques points intéressants.\\

Par analogie avec les maladies "virales", on peut distinguer les individus dits "sains", qui en l'occurence ne sont pas intéressés par le produit plubicisé ou qui n'en connaissent encore pas l'existence, des individus dits "infectés", qui utilisent le produit et qui sont donc suscpetibles d'influencer leur entourage à l'acheter également.
Il apparaît notamment que comme toute épidémie, n'importe quelle interaction entre une personne "infectée" et une personne "saine" peut résulter à une contamination. Cependant, il semble que la probabilité de propagation diminue à interactions répétées.
Par ailleurs, contrairement à une épidémie virale, le nombre de contacts entre une personne "infectée" avec une personne "saine" augmente la probabilité de contamination jusqu'à un certain seuil puis semble stagner. Ainsi, les individus semblent être imperméables aux recommandations de leur entourage si le produit ne les intéresse vraiment pas.

Comme évoqué précédemment, chaque individu possède une valeur de réseau qui indique l'influence de celui-ci sur les autres clients potentiels. Certains semblent très influents, ils sont appelés les \textbf{super-spreaders} \cite{viralTWeb} car ils "infectent" un grand nombre de personnes. Le modèle prend donc en compte qu'un super-spreader possède une plus forte probabilité de contamination. Néanmoins, il apparaît que lorsqu'un individu émet de plus en plus de recommandations, celles-ci s'avèrent de moins en moins efficaces.

Ce qui laisserait supposer qu'un super-spreader a beaucoup d'influence sur quelques membres de son entourage, mais pas sur toutes les personnes qu'il connaît.\\
Enfin, il semble que les produits les plus susceptibles de bénéficier d'une stratégie de marketing viral sont les produits technologiques ou idéologiques (religieux par exemple), puisqu'ils sont utilisés dans des contextes favorables comme une école, au travail ou sur un lieu de culte, des endroits ou tout un réseau est regroupé dans une même zone géographique et partage des centres d'intérêt communs.

Le marketing viral peut donc être très bénéfique puisque s'il fonctionne, le produit se répandra avec succès à moindre coup. Cependant c'est une stratégie risquée et limitée qui n'est pas adaptée à tous les produits et qui reste en grande partie abstraite malgré les nombreuses études sur le sujet.


\subsection{Ciblage des entités influentes}

La principale difficulté d'une stratégie de marketing virale est de cibler les individus ou entités les plus influents. La fouille de données semble être une méthode efficace pour déterminer qui est influent et qui possède un grand réseau. C'est en tout cas ce qu'indiquent les études de l'université de Cornell \cite{kdd03}.

Pour estimer la valeur du réseau d'un individu, une entreprise a besoin de connaître les relations entre les différents membres de son réseau. La source principale de ces informations est disponible sur Internet via les forums ou les réseaux sociaux. Les sites communautaires sont très souvent orientés sur les produits de consommation, il est donc possible de recueillir beaucoup d'informations sur les habitudes de consommation et les liens entre les différents membres de la communauté. On peut notamment trouver sur ces sites des tests complets de produits divers et variés, des évaluations, des classements, des comparaisons, des avis, des retours d'utilisation...

Supposons que nous disposons de données sur un réseau social, avec des estimations de l'influence mutuelle des individus, et que nous aimerions promouvoir un nouveau produit, qui nous l'espérons, sera adopté par une large partie de ce réseau.\\
Nous l'avons vu, le principe du marketing viral est qu'en ciblant d'abord quelques individus "influents" (par exemple, en leur donnant des échantillons gratuits des produits), il est possible de déclencher une cascade d'influence. Des amis recommandent le produit à d'autres amis, et ainsi de nombreux individus finiront par essayer. Mais comment doit-on choisir ces quelques individus clés (super-spreaders) à utiliser pour lancer ce processus au mieux ?

Une étude s'est fondée sur la fusion de deux algorithmes mathématiques imitant le comportement d'un réseau \cite{kdd02b}. Le premier est basique et linéaire, le second  prend en compte l'influence d'un membre sur les autres, il est dit "en cascade". Avec cette modélisation, il est possible de mesurer la dynamique d'un réseau de façon cyclique. À chaque cycle, un individu entre en contact avec son entourage et le contamine ou non.\\
Le problème réside alors dans le choix du noyau cible pour infecter le réseau le plus efficacement possible. Des chercheurs ont réfléchi à une fonction permettant de calculer pour chaque individu du réseau, en s'appuyant sur sa valeur de réseau, l'impact qu'il aura sur son entourage s'il est infecté \cite{MSNVM}. Cette fonction est ensuite optimisée afin de sélectionner un nombre défini de super-spreaders et en limitant les effets de redondance (par exemple en tentant de contaminer un individu déjà infecté). Ainsi est déterminé un noyau influent au sein d'un réseau, qui influence la grande majorité du réseau en quelques cycles seulement.

La limite à ses recherches est évidemment la qualité et la quantité des informations sur les relations entre les membres d'un réseau ainsi que l'évaluation de leur valeur de réseau. Cette technique de marketing est pour autant très prometteuse et continue à faire l'objet de nombreuses études.

La fouille de données s'avère donc être un outil puissant et très utilisé dans les stratégies de marketing actuelles. Appliquée à la mine d'or d'informations que représente aujourd'hui les réseaux sociaux et internet, cette science est largement exploitée par les entreprises. Celle-ci leur permet de cibler leur clientèle potentielle et de limiter leurs dépenses en prospection, en ne s'adressant qu'aux individus jugés susceptibles d'être demandeurs. C'est le fondement d'un nouveau type de marketing : le marketing viral. Il vise à cibler les personnes les plus influentes au sein d'un réseau afin de séduire un maximum d'individus en n'en démarchant qu'une petite partie. Ce marketing très prometteur fait l'objet de nombreuses études et semble se peaufiner, notamment dans la sélection du noyau d'individus influents.
