Les réseaux sociaux sont omniprésents depuis l'avènement d'Internet.\\
Ils permettent aux différents utilisateurs d'interagir en communauté et de se regrouper selon des critères qui leur sont importants.

Ces réseaux sociaux sont de différents types. Certains sont connus de tous (Facebook, Twitter, LinkedIn) et comptent des millions de membres. D'autres exploitent des niches moins connues et peuvent passer relativement inaperçus ou rester confidentiels, tels les réseaux d'entreprise.\\
Enfin, certains des échanges inter-personnels peuvent aussi être assimilés à des réseaux sociaux : c'est le cas des mails et des SMS, qui définissent des relations entre différents groupes d'individus.

Tous ces réseaux sociaux amassent de très nombreuses données : les amis, les messages, les images, la fréquence d'utilisation... tous ces échanges et informations sont soigneusement enregistrés. Dès lors se pose le problème de l'exploitation de cette masse d'informations.

Il faut tout d'abord modéliser le réseau sous forme mathématique. La structure de base est bien entendu le graphe : l'analyse des figures produites permet de tirer un grand nombre d'informations, et aussi de prédire en partie l'évolution future du réseau. Tous ces mécanismes seront abordés dans la partie "\l[#Les_processus_pour_la_fouille_de_données]{processus}".

Dès lors se pose la question de la vie privée. Les données récoltés et analysées, comme dit précédemment, permettent d'intuiter un grand nombre d'informations qu'un utilisateur ne souhaite pas forcément divulguer. Au coeur des débats se trouve bien entendu l'éthique de la fouille : des mécanismes sont donc mis en place pour protéger l'utilisateur de la curiosité parfois mal placée du fouilleur. Ces mécanismes sont détaillés dans la partie "\l[#Vie_privée_dans_les_réseaux_sociaux]{vie privée}".

La partie "\l[#Marketing_des_données]{marketing}" explicitera la façon dont ces informations sont rentabilisées en créant des stratégies marketing depuis les données précédemment analysées et protégées.
