La fouille de données s'avère être un outil incroyablement riche et puissant lorsqu'elle est appliquée aux réseaux sociaux.

La masse de données produites est riche et remplie d'informations pertinentes, ce qui permet aux processus de datamining de fonctionner correctement.\\
L'existence de très nombreuses recherches autour du sujet prouve, si besoin était, l'importance du phénomène et ses améliorations constantes.

N'oublions pas que ce domaine est extrêmement récent : pourtant, il capitalise déjà un nombre immense de publications et des avancées se font à un rythme presque quotidien.\\
Les outils mathématiques s'affûtent, gagnant en complexité afin de permettre de traiter des volumes de données toujours plus grands. Les théories de prédiction et d'analyse deviennent de plus en plus fiables et se démocratisent, permettant à tous de profiter de la richesse des informations contenues.

Mais si les scientifiques avancent à grands pas, les experts marketing ne sont pas en reste. Les théories et les avancées sont nombreuses dans ce domaine aussi : voilà quelques années, le concept même de marketing viral n'existait pas, alors qu'il est aujourd'hui devenu l'un des fers de lance de la communication par Internet.

Malheureusement, les problématiques de vie privées évoluent au moins aussi vite. La somme d'informations qu'un individu est prêt à mettre en ligne est presque sans limite, contrairement à sa volonté de les partager avec le monde. Échanger avec un cercle contrôlé, oui ; avec le monde entier ou le gouvernement, non. Ces attentes doivent être prises en compte par les différents réseaux lors de l'exploitation des données grâce aux différents mécanismes désormais mis en place.

Comme souvent, au vu de l'état de l'art dans le domaine, une grande question demeure : qu'en sera-t-il dans dix ans ? Quelle problématique l'emportera, les mathématiques, le marketing ou le repli face au public ? La question en elle-même manque de pertinence, tant les concepts présentés ici sont neufs et risquent d'évoluer en quelques mois seulement.

