Avec l'émergence des réseaux sociaux, de nouvelles problématiques propres à la vie privée sont soulevées. Nous verrons dans cette partie que des dangers, souvent négligés, existent pour les internautes. Par ailleurs, nous nous intéresserons aux procédés pour exploiter les données de ces utilisateurs sans leur porter atteinte.

\subsection{Lecture des données publiques}
Dans cette première partie, nous verrons ensemble qu'il est relativement facile de récupérer légalement des informations personnelles.

\subsubsection{Partage canapé et vie privée}
Des chercheurs se sont penchés sur le site \l[http://www.couchsurfing.org/]{CouchSurfing} dans le but de déterminer les informations que les internautes considéraient confidentielles \cite{couchsurfing}. Sur ce site, les utilisateurs proposent littéralement leur canapé à qui voudrait séjourner temporairement chez eux, et ce, gratuitement. Pour entrer dans le réseau, l'internaute est appelé à partager un grand nombre d'informations avec les autres membres. Les résultats de cette étude soulèvent un certain nombre de points intéressants. Les trois quarts des utilisateurs affichent clairement leur code postal, genre, âge mais aussi leur photo de profil, leur emploi et leur nom complet. Pour certains, cela va plus loin avec la date de naissance, e-mail, photos de leur maison, adresse complète ou encore numéro de téléphone.
Les raisons qui justifient de telles divulgations sont multiples. Aucun sondé ne semble particulièrement s'inquiéter par ces agissements. Certains évoquent le fait que ces mêmes informations sont accessibles dans l'annuaire téléphonique ou sur Google directement. Beaucoup reconnaissent divulguer des informations privées, mais ne voient pas l'intérêt qu'une personne pourrait avoir à les tracer et se doutent encore moins des exploitations malsaines, telles que l'usurpation d'identité. Mais le risque est bien là. Croiser ces informations avec d'autres sources peut permettre d'obtenir un profil très complet de la personne recherchée.

\subsubsection{Portrait volé}
À ce sujet, le magazine français \i{Le Tigre} possède une rubrique intitulée Portrait Google. Le principe est très simple : dresser le profil d'une personne inconnue via les informations qu'elle laisse sur internet. Fin 2008, le journaliste Raphaël Metz s'est lancé dans le cadre de cette rubrique à l'assaut d'un certain Marc L., une personne choisie au hasard \cite{marcL}. Après quelques recherches plus ou moins fructueuses sur les réseaux sociaux tels que Copains d'avant, Facebook, Flickr, Youtube, ou même Google, de nombreuses informations ont pu être récupérées. Tout cela lui a permis de rédiger une biographie relativement fournie en s'attachant à la fois aux relations amoureuses, expériences professionnelles, récits de vacances, ou encore coordonnées. Individuellement, comme dans l'étude précédente, les informations n'ont pas forcément une grande valeur. Mais une fois recoupées entre elles et en multipliant les sources, le résultat peut faire peur. En soit, la démarche n'a rien de scandaleuse, ces informations étant publiques. D'ailleurs, les recruteurs ont régulièrement recours à ces pratiques pour en savoir plus sur les personnes qu'ils s'apprêtent à engager. Le journaliste aurait même pu aller plus loin en s'inventant un personnage fictif cohérent avec le passé de sa cible, avec les mêmes intérêts ou en évoquant une rencontre passée imaginaire entre les deux dans un lieu identifié.
Cet article a créé la polémique dans les médias \cite{marcLFigaro}. En découvrant ce récit de sa vie, l'intéressé a fait supprimer toutes les informations le concernant sur internet et a demandé d'anonymiser l'article en question.

Ainsi, les internautes laissent globalement un grand nombre d'informations personnelles sur les réseaux sociaux, et internet plus généralement, sans se soucier trop des conséquences sur leur vie privée.

\subsection{Lecture des données privées}
Nous avons pu voir précédemment que beaucoup d'informations à caractère privé étaient visibles et donc accessibles à tous sur les réseaux sociaux. Maintenant, nous allons nous intéresser à la lecture et surtout l'exploitation des données privées.

\subsubsection{Google et la publicité}
En 1998, Google a lancé son célèbre moteur. Quelques années plus tard, ce même moteur de recherche se retrouve en position de monopole. Cette position a permis à la société de Montain View d'être au premier rang des attentes des utilisateurs, en voyant les requêtes que ces derniers formulaient. Il n'a pas fallu attendre très longtemps pour que ces informations soient exploitées. En effet, dès l'année 2000, soit seulement deux ans après la création de l'entreprise, Google lance AdWords. Ce système va permettre d'afficher de la publicité en rapport avec les recherches des internautes. Ce nouveau type de publicité présente un gros intérêt car elle se situe à l'endroit et au moment où l'utilisateur a le plus de chance de s'y intéresser. C'est le premier pas de la société de ce que l'on appelle maintenant la publicité ciblée. Depuis ce temps, cette activité représente la majeure partie des revenus du groupe. En 2010, la publicité a atteint 96% des 29 milliards de ses entrées d'argent \cite{revenusGoogle}.

\subsubsection{Ciblage indiscret}
Pour proposer une publicité de plus en plus ciblée, Google et les régies publicitaires ont besoin d'en savoir plus sur les internautes. Les cookies du navigateur internet sont d'ailleurs assez bavards à ce sujet. Mais cette recherche d'information sur les internautes est également le fil conducteur des services \i{gratuits} que Google lance. On peut notamment noter la suite bureautique \l[https://docs.google.com/]{Google Docs}, le partage de photos \l[http://picasa.google.fr]{Picasa} ou encore le courrier électronique \l[https://mail.google.com/]{GMail}. Ces services encouragent les internautes à déposer leurs informations personnelles directement sur les serveurs de Google. Ensuite, ce dernier peut analyser à souhait les données qu'il récupère, dans le but de proposer des publicités personnalisées. Ceci est particulièrement visible avec le service GMail où le courrier est automatiquement analysé par des \i{robots} \cite{pubGmail}. Par exemple, si un ami vous propose de l'accompagner à un séjour à la montagne, des offres pour des stations de ski apparaîtront dans la même fenêtre. Cette publicité est d'ailleurs très mal acceptée \cite{plainteGmail}. Le courrier, qu'il soit électronique ou papier, a toujours été perçu comme une ressource critique au niveau de la confidentialité. Le fait qu'une entreprise privée le lise ouvertement est perturbant vis-à-vis des libertés individuelles.
Pour compléter ces informations sur les internautes, Google doit se tourner vers une nouvelle source très riche : les réseaux sociaux.

\subsubsection{Réseaux sociaux : mine d'or du ciblage}
Leader incontesté des réseaux sociaux, Facebook compte aujourd'hui plus de 800 millions d'utilisateurs \cite{statsFB}. Très peu aurait imaginé qu'un jour plus d'un dixième de la population se soit créé un profil riche en informations personnelles sur un site internet privé. La monétisation de ses informations peut être très juteuse pour la jeune entreprise. Et pour cause ! Facebook propose aux annonceurs des options de ciblage extrême, comme les données démographiques, géographiques, ou encore les centres d'intérêt pour ne citer qu'eux. Il devient alors possible d'afficher des bannières aux personnes de la région parisienne, âgées de 26 à 30 ans et travaillant dans une entreprise concurrente. Quoi de mieux pour débaucher toute une équipe ?
Ceci n'est qu'une des nombreuses possibilités envisageables. Cette activité est d'autant plus intéressante pour Facebook que les informations personnelles sont fournies gratuitement et volontairement par les utilisateurs. C'est justement à ce niveau que Google a pris du retard \cite{pubReseauxSociaux}. Ce dernier a donc lancé l'été dernier son propre réseau social pour essayer de compléter son offre publicitaire. Mais il faut encore que le nombre d'utilisateurs atteigne une masse critique pour que la plate forme soit adoptée et fréquentée.

Ainsi, nous avons vu que les services gratuits sur internet étaient une mine d'or pour les régies publicitaires. Les réseaux sociaux sont particulièrement intéressants car ils fournissent \i{gratuitement} des informations pour un ciblage extrême et donc payant.

\subsubsection{Fouille de données dans le respect de vie privée}

Nous avons pu constater que la vie privée des individus pouvait être mis à mal sur internet, notamment à partir des réseaux sociaux. Pour essayer de préserver cette vie privée, différentes procédés ont été mis en place.

\subsubsection{Anonymisation délicate}
Simplement, nous pouvons penser qu'enlever les attributs tels que le nom et les coordonnées peut être suffisant mais ce n'est pas le cas. Comme nous l'avons vu précédemment, les données peuvent être recoupées et ainsi mettre en péril tout la vie privée d'une personne. Une femme de 62 ans a justement été identifiée et ses habitudes mises à nue suite à une erreur sur une base de données d'AOL, bien que son nom et autres informations critiques aient été masqués \cite{aolProfile}.
Nous pouvons également citer le cas du Netflix Prize Dataset. L'entreprise Netflix propose des services de location de films, films que ses clients peuvent noter. Pour améliorer leur service, la société a lancé un concours en mettant comme matière les notes enregistrées par presque un demi million de ses utilisateurs. Les informations d'identification avait préalablement été retirées, mais des chercheurs ont prouvé que cela n'était pas suffisant pour anonymyser l'ensemble \cite{deanonymization}. En effet, ils ont développé pour l'occasion une méthode pour \i{désanonymiser} les données de ce type de jeu de données\footnote{Très grande base avec des \i{micro-data} telles que les préférences sur un site web, les recommandations ou encore les transactions enregistrées}. Ils ont ainsi démontré que la connaissance de quelques informations sur un individu permettait de savoir s'il était présent dans l'échantillon et de le retrouver. Alors, il devient possible d'en déduire des informations telles que ses orientations politiques et sexuelles, à partir des notations de films qu'il a effectuées.
Nous allons donc nous pencher sur les méthodes reconnues pour travailler sur des données en préservant la vie privée.

\subsubsection{Ajout de bruit}
Une première approche consiste à faire varier de manière aléatoire les données recueillies \cite{randomDM}. Par exemple, dans le cas où l'étude cherche à analyser la distribution de l'âge des individus pour en établir une classification, il est possible d'ajouter une valeur de distorsion\footnote{Via une distribution uniforme sur un segment ou une distribution Gaussienne} à chacune des valeurs lors de l'acquisition des données. L'algorithme d'espérance-maximisation \cite{AlgoEM} est recommandé pour assurer la vraisemblance de l'ensemble collecté. Généralement, la fouille de données n'a pas vraiment besoin d'informations individuelles mais plus de distributions sur les propriétés étudiées. Pour cette raison, il est tout à fait envisageable de rajouter du bruit aléatoire et contrôlé sur les données tout en retrouvant la distribution initiale du groupe et donc en préservant l'objet de l'étude.
Cependant, cette approche n'est pas suffisante car elle permet de déduire certaines propriétés spécifiques du groupe d'étude. Il a été en effet démontré que l'ajout d'une petite perturbation aux données entraînait toujours une violation importante de la vie privée \cite{crackPrivacyDB}.

\subsubsection{Randomisation}
Pour remédier à ce problème, une approche a été entreprise en 2003 par des chercheurs de l'Université de Cornell et d'IBM pour compléter celle que nous venons d'évoquer \cite{privacyDM}. Celle-ci consiste à appliquer une \i{amplification} aux valeurs. Cette amplification consiste à utiliser un opérateur de \i{randomisation} qui vérifie certaines conditions. Cet opérateur diminue la probabilité de retrouver la valeur initiale à partir de la valeur initial et inversement.
L'intérêt de cette approche est qu'il n'est pas nécessaire d'avoir des connaissances ou de faire des hypothèses sur la distribution initiale. Par contre, il est difficile de quantifier son impact sur la précision de la fouille et donc de calibrer le paramétrage nécessaire à un certain degré respect de la vie privée.
Il s'agit d'un domaine très actif de la recherche, notamment à cause de l'effervescence des réseaux sociaux, les approches présentées ne sont donc pas exhaustives et d'autres sont encore en cours de développement.

Nous avons pu constater que des moyens plus ou moins efficaces existaient pour anonymiser les sources pour la fouille de données. Cela pourrait, à l'avenir, réduire le risque de divulgations d'informations sur les internautes. Encore faut-il que les entreprises prennent conscience des dangers de leurs activités et qu'elles fassent les efforts nécessaires pour y remédier.

Afin de conclure cette partie, les réseaux sociaux peuvent révéler des informations précieuses sur leurs utilisateurs. L'apprentissage sur ces derniers peut être d'autant plus grand que les sources sont multiples et croisées. De plus, les groupes qui récupèrent ces données ne se gênent pas pour les monétiser, sans forcément se soucier des risques de divulgation.
Des méthodes ont été développées pour justement essayer d'anonymiser ces données et ainsi limiter l'atteinte qu'une divulgation peut avoir sur les utilisateurs. Nous pouvons donc espérer que les risques diminuent à l'avenir.
